\documentclass[11pt, letterpaper]{article}

% Packages
\usepackage[margin=0.75in]{geometry}
\usepackage[utf8]{inputenc}
\usepackage[T1]{fontenc}
\usepackage{hyperref}
\usepackage{enumitem}
\usepackage{titlesec}
\usepackage{parskip}
\usepackage{xcolor}
\usepackage{fontawesome5}
\usepackage{graphicx}
\usepackage{lastpage}
\usepackage{fancyhdr}

\graphicspath{{gp_saggese_quant_research_cv/figures}}

% Hyperlink styling (HTML-style links)
\hypersetup{
  colorlinks=true,
  linkcolor=blue,
  urlcolor=blue,
  citecolor=blue,
}

% Page numbering setup
\pagestyle{fancy}
\fancyhf{} % Clear all header and footer fields
\fancyfoot[C]{\thepage/\pageref{LastPage}} % Center footer with page numbers
\renewcommand{\headrulewidth}{0pt} % Remove header line
\renewcommand{\footrulewidth}{0pt} % Remove footer line

% Set section numbering depth (0=none, 1=section, 2=subsection, 3=subsubsection)
\setcounter{secnumdepth}{1}

% Section formatting
\titleformat{\section}
{\Large\bfseries\filcenter}
{}
{0pt}
{\MakeUppercase}
[\titlerule]
\titlespacing{\section}{0pt}{1pt}{6pt}

\titleformat{\subsection}
{\large\bfseries}
{}
{0pt}
{}
\titlespacing{\subsection}{0pt}{8pt}{4pt}

% Reduce spacing in lists
\setlist{nosep, leftmargin=1.5em}
\setlist[itemize,1]{label=\textbullet}
\setlist[itemize,2]{label=\textopenbullet}

% Compact paragraphs
\setlength{\parindent}{0pt}
\setlength{\parskip}{6pt}

% Pandoc compatibility
\providecommand{\tightlist}{%
\setlength{\itemsep}{0pt}
\setlength{\parskip}{0pt}}

% Pandoc uses these for strikeout, etc.

% Code highlighting

% Tables

% Graphics

\begin{document}
  \begin{center}
    {\LARGE \textbf{Giacinto Paolo (GP) Saggese, PhD}} \\[6pt]

    {\large Email: \href{mailto:gp@causify.ai}{gp@causify.ai} \;|\; \href{mailto:gsaggese@umd.edu}{gsaggese@umd.edu} \;|\; \href{mailto:saggese@gmail.com}{saggese@gmail.com} }\\[6pt]

    {\large \href{https://www.linkedin.com/in/gpsaggese/}{LinkedIn} \;|\; \href{https://github.com/gpsaggese}{GitHub} \;|\; \href{https://gpsaggese.github.io/umd_classes/}{Website} }\\[6pt]

    {\large Cell: 408--431--1286, Washington DC }
  \end{center}

  \vspace{12pt}

  \begin{center}
    {\LARGE \textbf{Quant Research Statement}}
  \end{center}

  \vspace{12pt}

% Table of Contents
  \tableofcontents
  \newpage

% ==============================================================================
  \section{Executive Summary}
  \label{executive-summary}

  I am a quantitative researcher specializing in applying machine
  learning (ML) and artificial intelligence (AI) to financial problems, ranging
  from alpha-strategies to portfolio construction.

  My experience spans developing, trading, and evaluating over 100 alpha
  signals across asset classes (US equities, futures, crypto) and trading
  frequencies (from minutes to days), utilizing both market microstructure
  and alternative data sources.

  In leadership roles, I interviewed 50+ quantitative analysts, led research
  teams, and conducted thorough due diligence on alternative datasets.

  I developed proprietary research infrastructure including alpha signal
  backtesting frameworks, portfolio optimization systems, and market impact
  models.

  Since 2020, I teach graduate-level courses including ``DATA605:
  Big Data Systems'' and ``MSML610: Advanced Machine Learning'' at the
  University of Maryland, College Park (ranked 9th among public US
  universities), to over 800 students.

  My research at the University of Maryland focuses on Bayesian statistics and
  Causal AI, specifically developing methods to:

  \begin{itemize}
    \item Distinguish causal relationships from spurious correlations in
      financial time series, enabling models that capture true economic
      mechanisms rather than data-mining artifacts

    \item Quantify uncertainty in model predictions using Bayesian hierarchical
      models, providing principled confidence intervals for trading decisions

    \item Build knowledge graphs that encode economic relationships across
      asset classes, sectors, and macroeconomic factors

    \item Apply temporal machine learning to adapt to regime changes while
      maintaining interpretability of learned patterns
  \end{itemize}

  These methodologies have been successfully applied to financial prediction,
  predictive maintenance, demand forecasting, and energy trading, consistently
  improving both accuracy and robustness over traditional approaches.

  \textbf{Quantifiable Impact}:

  \begin{itemize}
    \item Contributed 25\% of profit to hedge funds managing \$1.6B in assets
      under management (AUM) across global futures and US equities

    \item Generated \$90M in profits with Sharpe ratio of approximately 3
      over 18 months (mid-2016 to end of 2017), achieving 56\% returns at
      2x volatility target

    \item Successfully deployed over 100 alpha signals in production across
      US equities, global futures, and ETFs, achieving Sharpe ratios ranging
      from 1.5 to 5.0 across different strategies

    \item Built and led research teams totaling 15+ quantitative researchers
      across two offices (Berkeley, CA and Moscow, Russia) that have integrated
      alternative data sources including news sentiment, social media,
      macroeconomic indicators, and SEC filings into quantitative strategies.

    \item Built two complete trading systems from data ingestion through
      portfolio construction.
  \end{itemize}

  My research philosophy centers on using modern AI and machine learning
  to complement---not replace---traditional quantitative methods,
  maintaining a high-throughput research process where models are
  discovered, validated, and deployed at a predictable rate while
  accounting for their limited lifespan in competitive markets.

  Complete CV and additional materials: \href{https://github.com/gpsaggese/umd_classes}{https://github.com/gpsaggese/umd\_classes}

  \begin{itemize}
    \item \href{https://gpsaggese.github.io/umd_classes/03_education/}{Education}

    \item \href{https://gpsaggese.github.io/umd_classes/05_publications/}{Publications}

    \item \href{https://gpsaggese.github.io/umd_classes/06_research/}{AI
      Research}
  \end{itemize}

% ==============================================================================
  \section{Research Philosophy}
  \label{research-philosophy}

  My research foundation in Causal AI and Bayesian Statistics at the
  University of Maryland, and applied work at Causify.AI, informs my
  approach to quantitative finance. I focus on:

  \begin{itemize}
    \item \textbf{Causal Inference}: Moving beyond correlation to understand
      causal relationships in financial markets, enabling more robust
      models that capture underlying economic mechanisms rather than
      spurious patterns

    \item \textbf{Bayesian Methods}: Incorporating prior knowledge and uncertainty
      quantification into model development. This allows principled
      handling of parameter uncertainty and model selection.

    \item \textbf{Decision-Making Under Uncertainty}: Combining probabilistic
      reasoning with temporal machine learning to build adaptive models.
      These models adjust to changing market regimes and account for
      epistemic uncertainty.
  \end{itemize}

  This foundation shapes my philosophy on alpha generation, where interpretability,
  economic reasoning, and rigorous statistical validation are paramount.

% ------------------------------------------------------------------------------
  \subsection{Core Principles for Alpha Generation}
  \label{core-principles-for-alpha-generation}

  Successful alpha generation requires:

  \begin{itemize}
    \item Models must be interpretable and grounded in economic theory

    \item State-of-the-art machine learning algorithms should complement
      traditional statistical methods

    \item I design high-throughput research processes that account for the
      limited lifespan of alpha signals in competitive markets, ensuring
      models are discovered and deployed in production at a predictable,
      steady rate

    \item Maximizing the impact of alternative data means investigating signals:

      \begin{itemize}
        \item At different time scales, ranging from minutes to days

        \item For multiple asset classes (futures, equities, ETFs, FX,
          etc.)

        \item For prediction of both factors and residuals

        \item For both continuous and event-driven trading
      \end{itemize}

    \item Ability to design and implement sophisticated and proprietary data
      pipelines is needed in addition to expertise in quant methodologies

    \item Combination of strong engineering practices and rigorous research
      and development (R\&D) accelerates research results
  \end{itemize}

% ==============================================================================
  \section{Applications}
  \label{applications}

% ------------------------------------------------------------------------------
  \subsection{Portfolio Optimization}
  \label{portfolio-optimization}

  My approach to portfolio construction emphasizes pragmatism and iterative
  improvement:

  \textbf{Construction Philosophy}:

  \begin{itemize}
    \item \textbf{Start Simple}: Begin with realistic portfolio optimization
      that respects trading constraints

    \item \textbf{Iterative Improvement}: Progressively enhance optimization
      to address signal peculiarities (e.g., sparsity, different time scales)

    \item \textbf{Leverage Existing Infrastructure}: Utilize available components
      for:

      \begin{itemize}
        \item Mean-variance optimization with standard constraints

        \item Covariance estimation

        \item Transaction cost models

        \item Market impact models
      \end{itemize}
  \end{itemize}

  \textbf{Portfolio Constraints and Management}:

  \begin{itemize}
    \item Market neutrality (primary concern)

    \item Low sector exposure

    \item Factor exposure controls (verified ex-post)

    \item Integration with Barra/Axioma factors for optimization
  \end{itemize}

  \textbf{Prediction Targets}:

  \begin{itemize}
    \item Optimize for raw returns (from 5 minutes to 1 day ahead)

    \item Mixer blends forecasts from all models

    \item Single-step and multi-period optimization approaches
  \end{itemize}

  \textbf{Signal Integration}:

  \begin{itemize}
    \item Alternative data signals are typically additive to microstructure,
      price-based, and fundamental signals due to low correlation

    \item Occasional challenges arise with strict factor exposure limits
      when alternative data predicts primarily a single factor

    \item Solution: Dedicated system for trading sectors/factors
  \end{itemize}

  \textbf{Execution Approach}:

  \begin{itemize}
    \item Leverage available infrastructure (market impact models, trading
      algos, fill simulators)

    \item Incorporate actual models into backtesting/optimization systems

    \item Account for costs and optimize for target trading backend

    \item API-based target portfolio specification for next period
  \end{itemize}

% ------------------------------------------------------------------------------
  \subsection{Alternative Data and Alpha Generation}
  \label{alternative-data-and-alpha-generation}

  Alternative data sets are prioritized based on:

  \begin{itemize}
    \item Large coverage of the trading universe

    \item Frequent updates

    \item Usable across different asset classes (e.g., equities and
      futures) to amortize research and acquisition costs
  \end{itemize}

  \textbf{Data Sets with Past Research Success}:

  \begin{itemize}
    \item News (e.g., news headlines or full-text using natural language
      processing)

    \item Social sentiment (e.g., from blogs, Twitter, StockTwits)

    \item Anonymized spending reports, credit card transactions, payroll
      data

    \item Macroeconomic data (e.g., for nowcasting inflation, industrial
      production, housing starts)

    \item Business reporting and disclosures

    \item Smart-phone app data for geolocation (e.g., for retail traffic)

    \item Supply chain information (e.g., trucking, rail freight data)

    \item Satellite data (e.g., for agriculture and ship tracking)

    \item Online retail prices (e.g., for inflation)

    \item Weather data
  \end{itemize}

  \textbf{Models Developed and Deployed} (with Sharpe Ratios):

  \begin{itemize}
    \item News sentiment models using Bloomberg, Reuters, Ravenpack (Sharpe
      ratio \textasciitilde1.5-2)

    \item Social sentiment from Twitter and StockTwits (Sharpe ratio
      \textasciitilde1.0-2.0)

    \item SEC insider trading signals (Sharpe ratio \textgreater1.5)

    \item SEC fundamental analysis with Compustat integration (Sharpe
      ratio \textgreater2.0)

    \item Value strategies combining Compustat and news (Sharpe ratio
      \textgreater2.0)

    \item Oil/natural gas models using news and 15-min futures (Sharpe
      ratio 3-5)

    \item Macroeconomic signals from supply chain datasets (Sharpe ratio
      \textasciitilde1-1.5)

    \item Spot foreign exchange (FX) models using purchasing power parity
      (Sharpe ratio \textasciitilde2-3)

    \item Short-Term Interest Rate (STIR)/bonds future arbitrage with treasury
      auction data (Sharpe ratio \textasciitilde1.5-2)

    \item Microstructure models using order book data (Sharpe ratio
      \textgreater3.0)
  \end{itemize}

% ------------------------------------------------------------------------------
  \subsection{Asset Classes and Trading Frequencies}
  \label{asset-classes-and-trading-frequencies}

  \textbf{Primary Targets}:

  \begin{itemize}
    \item US equities (market neutral, sector neutral, directional
      strategies)

    \item Global futures (outrights, calendar spreads for financials,
      energy, interest rates, currencies, commodities, agriculture)

    \item US exchange-traded funds (ETFs) for factor and sector prediction
      strategies
  \end{itemize}

  \textbf{Secondary Targets}:

  \begin{itemize}
    \item Spot foreign exchange

    \item Global equities (Japan, Europe)
  \end{itemize}

  \textbf{Trading Frequencies}:

  \begin{itemize}
    \item Mid-frequency strategies ranging from minutes to days

    \item Optimal frequency depends on the alpha models from specific alternative
      data sets and transaction cost estimates
  \end{itemize}

  \textbf{Frequency Examples by Data Type}:

  \begin{itemize}
    \item Many models from alternative data sets trade once per day

    \item Some data sets (news, social sentiment, blogs) can be traded at
      higher frequencies, from 15-minute intervals up to every minute

    \item Other data sets allow trading around specific events (earnings,
      mergers and acquisitions, executive departures, credit rating changes)
      on the time scale of hours, days, or weeks

    \item Some models combine market microstructure data with fast alternative
      data (e.g., news) to trade on minutely or hourly time scales
  \end{itemize}

% ==============================================================================
  \section{Research Methodology}
  \label{research-methodology}

  My approach to alpha-generation research is designed to combat overfitting
  while maintaining high research velocity. The methodology combines
  rigorous statistical controls with practical mechanisms to prevent
  common pitfalls in quantitative finance.

  I follow a structured process that prioritizes data integrity and
  statistical rigor:

  \begin{enumerate}
    \def\labelenumi{\arabic{enumi}.}

    \item \textbf{Data Isolation}: Separate holdout data immediately before
      any work begins to prevent information leakage

    \item \textbf{Exploratory Analysis}: Rigorously explore raw data without
      supervision from prices, understanding biases, collection processes,
      and data integrity

    \item \textbf{Literature Review}: Review context and generation of data
      sets while avoiding publication bias

    \item \textbf{Hypothesis Formulation}: Specify all hypotheses upfront,
      counting every parameter choice as a separate hypothesis

    \item \textbf{Feature Engineering}:

      \begin{itemize}
        \item Apply data transformations (e.g., differentiation, standardization,
          outlier detection) to extract information from data

        \item Dimensionality reduction using Principal Component Analysis
          (PCA) and Independent Component Analysis (ICA)

        \item Deseasonalization and residualization
      \end{itemize}

    \item \textbf{Backtesting}: Test hypotheses using cross-validation,
      regularization, and walk-forward testing

    \item \textbf{Robustness Testing}: Bootstrap analysis, sensitivity
      analysis, and clipping tests

    \item \textbf{Stationarity Testing}: Verify metrics remain consistent
      over time

    \item \textbf{Multiple Hypothesis Correction}: Apply Benjamini-Hochberg
      adjustment and White-reality check

    \item \textbf{Independent Review}: Double-check analysis and implementation
      with fresh eyes

    \item \textbf{Out-of-Sample (OOS) Validation}: Open OOS box only when
      no further changes are possible

    \item \textbf{Portfolio Integration}:

      \begin{itemize}
        \item Apply transaction cost models

        \item Test correlation with existing signals

        \item Mean-variance optimization
      \end{itemize}
  \end{enumerate}

% ------------------------------------------------------------------------------
  \subsection{Controlling Overfitting}
  \label{controlling-overfitting}

  This process combats overfitting through multiple layers of protection:

  \begin{itemize}
    \item \textbf{Statistical Controls}:

      \begin{itemize}
        \item In-sample cross-validation and OOS testing

        \item Prevent train/test information leakage to trust OOS results

        \item Multiple hypothesis testing corrections (Benjamini-Hochberg
          adjustment, White-reality check)

        \item Bootstrap and sensitivity analysis for robustness
      \end{itemize}

    \item \textbf{Process Discipline}:

      \begin{itemize}
        \item Specify and count all hypotheses upfront

        \item Do not HARK (Hypothesize After Results are Known)

        \item Maintain velocity to turn ideas into trading strategies quickly
      \end{itemize}

    \item \textbf{Practical Safeguards}:

      \begin{itemize}
        \item Implement mechanisms to avoid mistakes (e.g., future
          peaking, overly optimistic assumptions on market impact)

        \item Apply common sense to model validation and interpretation

        \item Independent review with fresh eyes
      \end{itemize}
  \end{itemize}

% ------------------------------------------------------------------------------
  \subsection{Machine Learning and AI Approach}
  \label{machine-learning-and-ai-approach}

  My philosophy on applying AI and ML to quantitative finance emphasizes
  complementing traditional methods rather than replacing them:

  \textbf{Core Principles}:

  \begin{itemize}
    \item Accelerate how research in finance is done (not change the
      basics of quant research)

    \item Extract forecasts from noisy and high-dimensional data sets

    \item Complement traditional statistical methods (not replace them)
  \end{itemize}

  \textbf{Implementation Strategy}:

  \begin{itemize}
    \item \textbf{Simple First}: Simple models should always be preferred
      over complex ones

    \item \textbf{Balance Complexity}: Strike a balance between model complexity
      and data to minimize overfitting

    \item \textbf{Strategic Application}: Deploy machine learning when:

      \begin{itemize}
        \item New classes of data become available

        \item Old alpha strategies are arbitraged away to extract more information

        \item Simpler models lack capacity to explain market reactions

        \item Widely copied strategies diminish headroom
      \end{itemize}
  \end{itemize}

  \textbf{Applications in Practice}:

  \begin{itemize}
    \item Feature extraction from alternative data sets

    \item Natural language processing (NLP) for news and social sentiment
      analysis

    \item Deep learning for text processing

    \item Dimensionality reduction and residualization

    \item Pattern recognition in complex, high-dimensional data
  \end{itemize}

  \textbf{Modern AI Technologies}:

  Recent advances in AI have opened new possibilities for quantitative finance,
  which I evaluate within my rigorous research framework:

  \begin{itemize}
    \item \textbf{Large Language Models (LLMs)}: Applying models like GPT
      and Claude for extracting nuanced information from financial text,
      earnings call transcripts, and regulatory filings. LLMs excel at
      understanding context and relationships that traditional NLP
      methods miss.

    \item \textbf{Transformer Architectures}: Using attention mechanisms
      for both time-series forecasting and text analysis. Transformers
      capture long-range dependencies in market data and can model
      complex temporal patterns.

    \item \textbf{Graph Neural Networks (GNNs)}: Leveraging GNNs to model
      knowledge graphs in causal inference, capturing relationships
      between companies, sectors, and economic factors. This aligns with
      my research on causal AI and enhances understanding of market interconnections.

    \item \textbf{Deep Learning Frameworks}: Utilizing PyTorch and TensorFlow
      for building custom architectures tailored to financial prediction
      tasks.
  \end{itemize}

  These modern techniques are most valuable when:

  \begin{itemize}
    \item Traditional methods cannot capture the complexity in the data

    \item Sufficient data is available to avoid overfitting

    \item The model remains interpretable or can be validated through rigorous
      testing

    \item The incremental benefit justifies the additional complexity
  \end{itemize}

  \textbf{Key Insight}: Traditional finance relies on heuristics and
  ordinary least squares (OLS) regression. ML offers tools to extract value
  that traditional methods cannot access, but must be applied rigorously
  within the disciplined research framework to avoid overfitting in
  financial markets' low signal-to-noise environment.

% ==============================================================================
  \section{Technology and Infrastructure}
  \label{technology-and-infrastructure}

  I have built comprehensive trading infrastructure to support the
  research process and production trading.

  \textbf{Core Infrastructure Components}:

  \begin{itemize}
    \item Toolkit for exploratory analysis of alternative data

    \item Research pipeline (feature computation, hypothesis testing,
      cross-validation)

    \item Event study framework for event-driven trading

    \item Backtesting system for multiple asset classes

    \item Mean-variance optimizer with constraints (cvxopt, numpy)

    \item Portfolio attribution and analysis tools

    \item Market impact models

    \item Real-time data processing pipelines

    \item Automated model deployment system
  \end{itemize}

  \textbf{Technology Stack}:

  \emph{Core Platform}:

  \begin{itemize}
    \item Linux-based infrastructure with Python as primary language

    \item Scientific computing: NumPy, pandas, SciPy for numerical
      analysis

    \item Traditional ML: scikit-learn for classical machine learning
      algorithms

    \item Deep learning: PyTorch and TensorFlow for neural network
      architectures

    \item Optimization: cvxopt for portfolio optimization, custom solvers
      for constraints
  \end{itemize}

  \emph{Data and ML Operations}:

  \begin{itemize}
    \item Cloud infrastructure: Amazon AWS (EC2, S3, RDS) for scalable
      compute and storage

    \item Data pipelines: Real-time and batch processing for market and
      alternative data

    \item Version control: Git for code, DVC for data versioning

    \item MLOps: Model versioning, automated retraining, and deployment
      pipelines

    \item Monitoring: Real-time model performance tracking and alerting
      systems
  \end{itemize}

  \emph{Integration and Deployment}:

  \begin{itemize}
    \item API integrations for market data (Bloomberg, Reuters), fundamental
      data (Compustat), and alternative data sources

    \item Order management system (OMS) integration for trade execution

    \item Database systems: PostgreSQL for structured data, time-series databases
      for market data

    \item Two complete end-to-end trading systems built from scratch (data
      ingestion through portfolio construction)
  \end{itemize}

% ==============================================================================
  \section{Example of Trading Research in 2015-2019}
  \label{example-of-trading-research-in-2015-2019}

  The following example illustrates the application of my research
  methodology in practice, covering the period from 2015 to 2019.

% ------------------------------------------------------------------------------
  \subsection{2015: Building the Foundation}
  \label{building-the-foundation}

  \textbf{Context}:

  \begin{itemize}
    \item Joined as the 2nd person in the alternative data group in Berkeley

    \item No existing research pipeline for ``low frequency'' trading,
      no data APIs

    \item Everything was geared towards high-frequency trading (HFT) with
      a pervasive assumption that nothing was held overnight

    \item Initial focus: futures markets
  \end{itemize}

  \textbf{Futures Trading Scope:}

  \begin{itemize}
    \item 80 futures contracts across US, Europe, and Asia with varying liquidity
      levels

      \begin{itemize}
        \item Asset classes: equity, fixed income, currency, commodities,
          energies

        \item Instruments: outrights, calendar spreads, butterflies (curvature
          of term structure)

        \item Trading frequencies:

          \begin{itemize}
            \item 20 contracts: Time-Weighted Average Price (TWAP) to
              TWAP (trade over the day)

            \item 13/19 contracts: 5 minutes

            \item Intraday: around open, around close, open/mid-day (good
              results but not traded due to operational complexity and redundancy)
          \end{itemize}
      \end{itemize}
  \end{itemize}

% ------------------------------------------------------------------------------
  \subsection{2016-2017: Scaling Up}
  \label{scaling-up}

  \textbf{By mid-2016}:

  \begin{itemize}
    \item Deployed \textasciitilde40 models across multiple strategies:

      \begin{itemize}
        \item Macroeconomic

        \item Alternative data

        \item Microstructure

        \item Price-volume (Commodity Trading Advisor-like)
      \end{itemize}

    \item Several alternative data models from 4-5 datasets online:

      \begin{itemize}
        \item Macroecon (GDP estimates tracking trains and ships)

        \item Oil models using oil production and temperature data

        \item News sentiment

        \item Social media sentiment

        \item Microstructure and price-based futures models
      \end{itemize}

    \item Research hit rate \textasciitilde50\% (2-3 data sets didn't
      pan out)

    \item Historical Sharpe ratio \textasciitilde4-5 for futures, before
      costs
  \end{itemize}

  \textbf{Teza Futures Returns (Jan 2016 - Jan 2017):}

  \begin{itemize}
    \item Overall: Up a few percent (moderate performance)

    \item Alternative data signals: Performed well
  \end{itemize}

% ------------------------------------------------------------------------------
  \subsection{2017-2018: Peak Performance and Transition}
  \label{peak-performance-and-transition}

  \textbf{2017 Performance}:

  \begin{itemize}
    \item First year when the futures system was fully completed

    \item \$1B assets under management (AUM)

    \item \textasciitilde25\% of risk allocated to alternative data

    \item Mid-2016 to end of 2017: Sharpe ratio \textasciitilde3, \$90M profit
      and loss (PnL)

    \item 2x volatility target → returns \textasciitilde56\%
  \end{itemize}

  \textbf{Strategic Shift:}

  \begin{itemize}
    \tightlist

    \item After 2017, started focusing on equities, FX, and ETFs
  \end{itemize}

  \textbf{Company Restructuring:}

  \begin{itemize}
    \item Significant organizational changes:

      \begin{itemize}
        \item Leadership departures: Chief Investment Officer (CIO), Chief
          Risk Officer (CRO), and 2 Chief Technology Officers (CTOs)
          left

        \item HFT business sold to QuantLab (70 people plus substantial
          core infrastructure)

        \item Previously working infrastructure had to be reimplemented from
          scratch
      \end{itemize}
  \end{itemize}

% ------------------------------------------------------------------------------
  \subsection{2018-2019: Challenges and Resilience}
  \label{challenges-and-resilience}

  \textbf{Trading Challenges}:

  \begin{itemize}
    \item Feb 2018: Significant drawdown in futures (10\% loss in one
      month)

      \begin{itemize}
        \item Mean-reversion models started losing money simultaneously

        \item Risk control failure: systems that had been performing
          well received increased weight, amplifying losses
      \end{itemize}

    \item New portfolio manager (PM) brought in for futures

      \begin{itemize}
        \item Rebuilt portfolio optimization from scratch using already-researched
          signals

        \item Started adding signals one at a time with overfitting checks

        \item Sept 2018: Began adding back alternative data signals
      \end{itemize}
  \end{itemize}

  \textbf{Alternative Data Signal Deployment Issues:}

  \begin{itemize}
    \item All alternative data signals (both old and new) were not traded
      in 2018

      \begin{itemize}
        \item Affected both equities and futures

        \item When signals were eventually put in production, backtests
          showed they would have performed well
      \end{itemize}
  \end{itemize}

  \textbf{Research Output:}

  In 2018, the alternative data team released \textasciitilde10 signals from
  4-5 data sets:

  \begin{itemize}
    \item Compustat ``latency arbitrage'' (SEC data, proprietary)

    \item Macro-econ news for futures (RavenPack)

    \item Multiple news-related signals (both event-driven and
      continuous, RavenPack)

    \item FX signals estimating Purchasing Power Parity (Numbeo)

    \item Two social sentiment signals (PsychSignal and Social Market
      Analytics)

    \item Auditor change signals

    \item Company disclosure signals
  \end{itemize}

% ==============================================================================
  \section{Example of Quant Strategies}
  \label{example-of-quant-strategies}

% ------------------------------------------------------------------------------
  \subsection{US Equity: Daily}
  \label{us-equity-daily}

  \begin{itemize}
    \item Frequency: multiple events per day

    \item 6 different models (e.g., earning announcement, stat arb, book
      pressure)
  \end{itemize}

  Figure 1 shows the cumulative performance of daily US equity strategies,
  demonstrating consistent returns across multiple event-driven models with
  strong risk-adjusted performance.

  \begin{center}
    \includegraphics[width=6.5in]{image8.png}

    \textit{Figure 1: Daily US equity strategy cumulative performance}
  \end{center}

% ------------------------------------------------------------------------------
  \subsection{US Equity: Intraday}
  \label{us-equity-intraday}

  \begin{itemize}
    \item Universe

      \begin{itemize}
        \tightlist

        \item 600 most liquid US equities, updated monthly
      \end{itemize}

    \item Microstructure-based

      \begin{itemize}
        \tightlist

        \item Machine learning on order book data
      \end{itemize}

    \item Trade 5 to 30 minute waves

      \begin{itemize}
        \item With or without overnight holdings

        \item Static feature weights derived independently from price data
      \end{itemize}

    \item Execution

      \begin{itemize}
        \tightlist

        \item Mostly passive/market making TWAP/VWAP
      \end{itemize}

    \item Sharpe ratio: \textgreater5

    \item Capacity: \$10-50M
  \end{itemize}

  Figures 2-3 illustrate the intraday US equity strategy performance,
  showing cumulative returns and drawdown characteristics for high-frequency
  microstructure-based models achieving Sharpe ratios exceeding 5.

  \begin{center}
    \includegraphics[width=4.48in]{image6.png}
    \includegraphics[width=4.40in]{image7.png}

    \textit{Figures 2-3: Intraday equity strategy performance and drawdowns}
  \end{center}

% ------------------------------------------------------------------------------
  \subsection{Alternative-data for US Equities / Futures}
  \label{alternative-data-for-us-equities-futures}

  Between 2015 and 2019 I was head of data at Teza. My main duty was to
  produce alpha strategies for Teza funds. I've supervised a team of 5 researchers
  in Berkeley, CA and around 10 researchers / developers in the Moscow, Russia
  office.

  Teza ran two funds from external investors:

  \begin{itemize}
    \item Global futures (around \$1b)

    \item US equities (around \$600m)
  \end{itemize}

  My team contributed around 25\% of PnL to both funds.

  Figure 4 presents the realized performance of the global futures fund
  from its 2015 launch, demonstrating consistent profitability with Sharpe
  ratio of approximately 3 during peak performance periods (2016-2017).

  \begin{center}
    \includegraphics[width=6.5in]{image3.png}

    \textit{Figure 4: Global futures fund realized performance (2015-2019)}
  \end{center}

  Figures 5-6 show the US equity fund performance launched in 2017,
  illustrating returns and risk metrics across different market conditions.

  \begin{center}
    \includegraphics[width=6.5in]{image2.png}

    \textit{Figure 5: US equity fund performance (2017-2019)}
  \end{center}

  \begin{center}
    \includegraphics[width=6.5in]{image1.png}

    \textit{Figure 6: US equity fund risk metrics and returns}
  \end{center}

  Figure 7 details the composition and diversification across strategy types,
  showing the contribution of alternative data models alongside microstructure
  and fundamental strategies.

  \begin{center}
    \includegraphics[width=6.5in]{image5.png}

    \textit{Figure 7: Strategy composition and diversification}
  \end{center}

  External validation of this work appeared in the Wall Street Journal:

  From \href{https://www.wsj.com/articles/the-future-is-bumpy-high-tech-hedge-fund-hits-limits-of-robot-stock-picking-1513007557}{The
  Future Is Bumpy: High-Tech Hedge Fund Hits Limits of Robot Stock Picking
  - WSJ}:

  ``Teza Capital Management LLC credits machine learning in part for its
  more than 50\% gain so far this year.''

% ------------------------------------------------------------------------------
  \subsection{Alpha research on Crypto statarb}
  \label{alpha-research-on-crypto-statarb}

  An example of alpha HFT strategies for crypto is below.

  \begin{itemize}
    \item \emph{Instrument type}: Perpetual futures 25 crypto

    \item \emph{Universe}: Most liquid 25 crypto tokens (excluding
      stable coins)

      \begin{itemize}
        \item E.g., ETH, BTC, SAND, STORJ, GMT, AVAX, BNB, APE, MATIC,
          DOT, UNFI, LINK, XRP, RUNE, NEAR, FTM, WAVES, AXS, OGN, DOGE, SOL

        \item Total daily trading volume: Around \$1.2 billion
      \end{itemize}

    \item \emph{Exchange}: Binance

    \item \emph{Performance}: Figure 8 displays the cumulative returns of
      the cryptocurrency statistical arbitrage strategy, achieving a Sharpe
      ratio of approximately 7 with 150\% annualized returns since inception
      in January 2021.

      \begin{center}
        \includegraphics[width=6.24in]{image4.png}

        \textit{Figure 8: Cryptocurrency statistical arbitrage cumulative returns}
      \end{center}

      \begin{itemize}
        \item \emph{Number of trades per month}: 216,000 (=25 cryptos * 12
          trades per hour * 24 hours * 30 days)

        \item \emph{Annualized return}: 150\% / yr since inception (Jan 2021)

        \item \emph{YTD returns}: 310\% / yr, with a return of \$1,947,168,
          given a target Gross Market Value (GMV) of \$750,000 and
          target daily dollar volatility of \$150,000

        \item \emph{Sharpe ratio} (measure of risk-adjusted returns): \textasciitilde7

        \item \emph{Risk characteristics}: market neutral within 5\%, less
          than 20\% concentration in each token, the longest drawdown was
          14 days
      \end{itemize}

    \item \emph{Basic economic principle}: use market microstructure and
      order book-based signals to predict price movement on a short-time
      timeframe (ranging from minutes to hours)

    \item \emph{Holding period}: around 30 mins

      \begin{itemize}
        \tightlist

        \item We can liquidate the entire book within 15 mins
      \end{itemize}

    \item \emph{Brief description}: Every 5 to 15 mins we run our own proprietary
      statistical arbitrage model. We then find the optimal portfolio
      using our proprietary Bayesian mean-variance optimization, given current
      holdings, ideal holdings as predicted by the model, cost to execute,
      and risk in terms of the covariance matrix. We impose various constraints
      (e.g., market neutrality within 5\%, less than 20\% concentration in
      each token). Finally, we place trades through Binance.
  \end{itemize}
\end{document}
