%----------------------------------------------------------------------------------------
%	PACKAGES AND THEMES
%----------------------------------------------------------------------------------------

\documentclass[aspectratio=169, xcolor=dvipsnames]{beamer}
\usetheme{SimplePlus}

\usepackage{hyperref}
\usepackage{graphicx} % Allows including images
\usepackage{
  booktabs
} % Allows the use of \toprule, \midrule and \bottomrule in tables

\input{latex_abbrevs.sty}

%----------------------------------------------------------------------------------------
%	TITLE PAGE
%----------------------------------------------------------------------------------------

\title[short title]{Simple Beamer Theme}
% The short title appears at the bottom of every slide, the full title is only on the title page
\subtitle{Subtitle}

%\author{[]
%]
%{Dr. GP Saggese}

\institute[UMD]
% Your institution as it will appear on the bottom of every slide, may be shorthand to save space
{ Department of Computer Science\\ University of Maryland, College Park% Your institution for the title page
}
\date{\today} % Date, can be changed to a custom date

%----------------------------------------------------------------------------------------
%	PRESENTATION SLIDES
%----------------------------------------------------------------------------------------

\begin{document}
  \begin{frame}{Overview}
    % Throughout your presentation, if you choose to use \section{} and \subsection{} commands, these will automatically be printed on this slide as an overview of your presentation
    \tableofcontents
  \end{frame}

  % #######################################################################################
  \section{Map of the class}
  % #######################################################################################

  % ---------------------------------------------------------------------------------------
  \subsection{Books of the class}

  % ---------------------------------------------------------------------------------------
  \subsection{Invariants of the class}

  % #######################################################################################
  \section{What is machine learning?}
  % #######################################################################################

  \subsection{Machine Learning Concepts}

  \begin{frame}{Definition of Machine Learning}
    \begin{itemize}
      \item \textbf{Arthur Samuel (1959):} Machine learning is the study
        enabling computers to \textcolor{blue}{learn without explicit
        programming}

      \item Example: A computer learning to play checkers by \textit{memorizing
        winning positions}

      \item \textbf{Tom Mitchell (1998):} A program learns from experience $E$, with
        task $T$ and performance $P$ if \textcolor{purple}{$P(T)$ improves with
        $E$}
    \end{itemize}
  \end{frame}

  \begin{frame}{Assumptions in Machine Learning}
    \begin{itemize}
      \item \textbf{Key assumptions:}
        \begin{itemize}
          \item Patterns exist in data

          \item Mathematical pinning of patterns is challenging

          \item There is data
        \end{itemize}

      \item \textbf{Which assumption is key?}
        \begin{itemize}
          \item If there is no pattern we can try learning, measure if learning
            works and, in the worst case, conclude that it does not work

          \item If we can find the solution in one step or program the solution,
            machine learning is not the recommended technique, but it still works

          \item Without data we cannot do anything: data is all that matters

          \item \textcolor{red}{Data is the most crucial aspect}; without it,
            learning is infeasible
        \end{itemize}
    \end{itemize}
  \end{frame}

  \begin{frame}{Algorithm vs Data}
    \begin{itemize}
      \item The best ML algorithm is \textit{not always the winner}

      \item \textcolor{cyan}{Data quantity often outweighs algorithm quality}
    \end{itemize}
  \end{frame}

  \begin{frame}{Some ML adages}
    \begin{itemize}
      \item "Every time I fire a linguist, the performance of the speech recognizer
        goes up" (Researcher in speech recognition)

      \item "If you torture the data long enough, it will confess whatever you want"
        (data mining)

      \item "An explanation of the data should be as simple as possible, but not
        simpler" [Einstein] (overfitting)

      \item The simplest model that fits the data is also the most plausible [Occam’s
        razor]

      \item "Combination of data and desire for an answer does not ensure that a
        reasonable answer can extracted from the data" (John Tukey)
    \end{itemize}
  \end{frame}

  \begin{frame}{A map of machine learning}
    \begin{itemize}
      \item ML theory

      \item Paradigms

      \item Models

      \item Techniques
    \end{itemize}
  \end{frame}

  %```mermaid
  %mindmap
  %  root((A map of machine learning))
  %
  %    Theory
  %      sub-branch(VC theory)
  %      sub-branch(Bias-variance decomposition)
  %      sub-branch(Computation complexity)
  %      sub-branch(Bayesian)
  %
  %    Paradigms
  %      Supervised_learning
  %      Unsupervised_learning
  %      Reinforcement_learning
  %      Active_learning
  %      Online_learning
  %
  %    Models
  %      sub-branch(Linear models)
  %      sub-branch(Generalized linear models)
  %      sub-branch(Neural networks)
  %      sub-branch(SVM)
  %      sub-branch(Nearest neighbors)
  %        sub-branch(Clustering)
  %        sub-branch(KNN)
  %      sub-branch(Gaussian processes)
  %      sub-branch(Factor-based analysis)
  %      sub-branch(Graphical models)
  %        sub-branch(Model joint probability distributions using graphs)
  %        sub-branch(Hidden Markov models)
  %
  %    Techniques
  %      Input_processing
  %        sub-branch(Data cleaning)
  %        sub-branch(Dimensionality reduction)
  %        sub-branch(Feature engineering)
  %      Model_building
  %        sub-branch(Models)
  %        sub-branch(Learning algorithms)
  %      Performance evaluation
  %        sub-branch(Cross-validation)
  %        sub-branch(Bias / variance curves)
  %        sub-branch(Learning curves)
  %      Regularization
  %        sub-branch(Aggregation)
  %        sub-branch(Boosting)
  %        sub-branch(Bagging)
  %        sub-branch(Stacking)
  %```

  \begin{frame}{ML theory}
    \begin{itemize}
      \item The problem with ML theory is that assumptions are sometimes divorced
        from the practice problems

      \item VC theory: it can be applied in practice

      \item Bias-variance decomposition

      \item Computation complexity (e.g., MDL)

      \item Bayesian: ML as a branch of probability. If you have the joint probability
        distribution you can answer almost any question
    \end{itemize}
  \end{frame}

  \begin{frame}{ML paradigms}
    \begin{itemize}
      \item Many different set-ups for machine learning problems

      \item Supervised learning
        \begin{itemize}
          \item One has inputs and corresponding output
        \end{itemize}

      \item Unsupervised learning
        \begin{itemize}
          \item Unlabeled data

          \item Need to find structure in the data
        \end{itemize}

      \item Reinforcement learning
        \begin{itemize}
          \item Not have the correct answer immediately

          \item Need to judge an action by the final results
        \end{itemize}

      \item Active learning
        \begin{itemize}
          \item Not have all the examples beforehand

          \item Query the output for certain inputs
        \end{itemize}

      \item On-line learning
        \begin{itemize}
          \item Cannot store all the data set

          \item The data set is streamed and one tries to change the hypothesis on-the-flight
        \end{itemize}

      \item \ldots{}
    \end{itemize}
  \end{frame}

  \begin{frame}{Models}
    \begin{itemize}
      \item Linear models

      \item Generalized linear models
        \begin{itemize}
          \item Logistic regression

          \item Poisson regression
        \end{itemize}

      \item Neural networks

      \item SVM

      \item Nearest neighbors
        \begin{itemize}
          \item Clustering k-means

          \item KNN
        \end{itemize}

      \item Gaussian processes

      \item Graphical models: model joint probability distributions using graphs
        \begin{itemize}
          \item E.g., hidden Markov models
        \end{itemize}

      \item \ldots{}
    \end{itemize}
  \end{frame}

  \begin{frame}{ML techniques}
    \begin{itemize}
      \item Input processing
        \begin{itemize}
          \item Data cleaning

          \item Dimensionality reduction

          \item Feature engineering
        \end{itemize}

      \item Model building
        \begin{itemize}
          \item Models

          \item Learning algorithms
        \end{itemize}

      \item Performance evaluation
        \begin{itemize}
          \item Cross-validation

          \item Bias / variance curves

          \item Learning curves
        \end{itemize}

      \item Regularization
        \begin{itemize}
          \item To deal with overfitting
        \end{itemize}

      \item Aggregation
        \begin{itemize}
          \item Boosting

          \item Bagging

          \item Stacking
        \end{itemize}

      \item \ldots{}
    \end{itemize}
  \end{frame}

  \begin{frame}{Names for inputs and outputs}
    \begin{itemize}
      \item Inputs: predictor vars, independent vars, features

      \item Outputs: predicted vars, dependent vars, responses, targets

      \item Inputs and outputs can be of 3 types:
        \begin{itemize}
          \item Quantitative: continuous values

          \item Qualitative (aka categorical): discrete values
            \begin{itemize}
              \item E.g., handwritten letters recognition, species of Iris, success
                / failure, survived / died
            \end{itemize}

          \item Ordered categorical: categorical with an ordering, but without a
            distance between them
            \begin{itemize}
              \item E.g., small, medium, large
            \end{itemize}
        \end{itemize}
    \end{itemize}
  \end{frame}

  \begin{frame}{Supervised learning}
    \begin{itemize}
      \item One of the most commonly used paradigms of machine learning

      \item Unknown target function $y = f(\vx)$
        \begin{itemize}
          \item Regression if the predicted variable $y \in \bbR$

          \item Classification if $y \in \{y_{1}, ..., y_{k}\}$
        \end{itemize}

      \item Data set $(\vx_{1}, y_{1}), ... , (\vx_{n}, y_{n})$
        \begin{itemize}
          \item We know the (unknown) function on certain points
        \end{itemize}

      \item Learning algorithm picks $g \approx f$ from a hypothesis set $\calH$
        (read ``script h'')
    \end{itemize}
  \end{frame}

  \begin{frame}{Examples of supervised learning problems}
    \begin{itemize}
      \item Predict the price of a house given various features
        \begin{itemize}
          \item Features: size in square feet, number of bedrooms, age of home, zip
            code, \ldots{}
        \end{itemize}

      \item Predict credit line given features
        \begin{itemize}
          \item Features: salary, credit history, debt, years in job, \ldots{}
        \end{itemize}

      \item Predict malignant or benign cancer given features
        \begin{itemize}
          \item Features: tumor size, age, gender, \ldots{}
        \end{itemize}

      \item Sometimes the same problem can be cast as regression or classification

      \item E.g., approve credit line (yes/no) vs determine the credit line (dollar
        amount)
        \begin{itemize}
          \item Features: current debt, age, annual salary, years in job, \ldots{}
        \end{itemize}
    \end{itemize}
  \end{frame}

  \begin{frame}{Unsupervised learning}
  \begin{itemize}
  \item
    Data is un-labeled
  \item
    We want to find structure in the data (e.g., clustering, anomaly
    detection)
  \item
    E.g.,
    \begin{itemize}
    \item
      Clustering of similar news stories
    \item
      Anomaly detection (aircraft engine, account hacked based on typical
      user behavior, computer going to fail, \ldots)
    \item
      Market segmentation
    \item
      Social network analysis (e.g., find cliques representing close
      friends)
    \end{itemize}
  \end{itemize}
  \end{frame}

  % #######################################################################################
  \subsection{Practical ML}
  % #######################################################################################

  \begin{frame}{Machine learning research flow}
    \begin{itemize}
      \item Question
      \item Input data
      \item Features
      \item Algorithm
      \item Parameters
      \item Evaluation (benchmark)
    \end{itemize}
  \end{frame}

  %flowchart LR
  %    A(Question) --> B(Input data)
  %    B --> C(Features)
  %    C --> D(Algorithm)
  %    D --> E(Parameters)
  %    E --> F(Evaluation)

  \begin{frame}{Question}
    \begin{itemize}
      \item Question needs to be as concrete and precise as possible

      \item Question $>$ data $>$ features $>$ algorithm
    \end{itemize}
  \end{frame}

  \begin{frame}{Data}
    \begin{itemize}
      \item Data should be on exactly the same thing one is trying to predict
        \begin{itemize}
          \item E.g., to predict new movie ratings from old movie ratings
        \end{itemize}

      \item Sometimes the relationship is not as direct
        \begin{itemize}
          \item E.g., we are interested in prices, but we predict supply and
            demand
        \end{itemize}

      \item If the data is not good, then one ends with ``garbage in - garbage out''

      \item One needs to know when to give up when the data does not allow an answer

      \item More data \textgreater{} better models
        \begin{itemize}
          \item Meta-studies have shown that the difference between the generic model
            and the best model for a problem improves like 5\%
        \end{itemize}
    \end{itemize}
  \end{frame}

  \begin{frame}{Properties of a good feature}
    \begin{enumerate}
      \def\labelenumi{\arabic{enumi}.}

      \item Lead to data compression

      \item Retain relevant information

      \item Use expert knowledge

      \item Is interpretable
    \end{enumerate}
  \end{frame}

  \begin{frame}{Common mistakes when building features}
    \begin{enumerate}
      \def\labelenumi{\arabic{enumi}.}

      \item Trying to automate feature selection leads to overfitting
        \begin{itemize}
          \item Black box prediction can be very accurate

          \item System performance can change suddenly and we might not understand
            why
            \begin{itemize}
              \item E.g., Google flu people didn't understand exactly the link between
                features and model
            \end{itemize}
        \end{itemize}

      \item Not paying attention to data-specific quirks
        \begin{itemize}
          \item E.g., outliers that are not really outliers
        \end{itemize}

      \item Throwing away information unnecessarily
    \end{enumerate}
  \end{frame}

  \begin{frame}{Characteristics of best models}
    \begin{enumerate}
      \def\labelenumi{\arabic{enumi}.}

      \item Interpretable
        \begin{itemize}
          \item E.g., decision trees are a appropriate in medical studies since they
            produce a ``reasoning''
        \end{itemize}

      \item Simple

      \item Accurate
        \begin{itemize}
          \item Often accuracy is traded off for remaining characteristics

          \item E.g., accuracy vs interpretability, accuracy vs speed
        \end{itemize}

      \item Fast (to train and test)

      \item Scalable
        \begin{itemize}
          \item E.g., in the Netflix prize, Netflix didn't ended up implementing
            the best algorithm since it wasn't scalable enough
        \end{itemize}
    \end{enumerate}
  \end{frame}

\end{document}
