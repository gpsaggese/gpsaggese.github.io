% ##############################################################################
\section{Implications and Applications}
\label{sec:implications}

% ==============================================================================
\subsection{Moral and Philosophical Implications}

The framework clarifies debates about desert and responsibility. If formal
analysis demonstrates that outcomes depend substantially on luck rather than
controllable factors, several philosophical implications follow:

\begin{itemize}
  \item \textbf{Desert:} Claims that individuals fully deserve their
    outcomes become difficult to sustain when luck plays a significant
    role.

  \item \textbf{Responsibility:} Moral evaluation must account for the
    limited control agents have over their circumstances and the events
    that befall them.

  \item \textbf{Dignity:} If outcomes are substantially luck-determined,
    human worth should not be closely tied to achievement.
\end{itemize}

These implications align with philosophical analyses of moral luck,
which argue that factors beyond an agent's control affect our moral
judgments inappropriately.

% ==============================================================================
\subsection{Policy Implications}

Quantifying the role of luck informs policy debates across multiple domains.

% ------------------------------------------------------------------------------
\subsubsection{Taxation and Redistribution}

When luck contributes substantially to income and wealth, the case for progressive
taxation and redistribution strengthens. If high earners benefit from
favorable circumstances and chance events, claims of full entitlement to
their income weaken. Luck-adjusted taxation could incorporate
measurements of the luck component in individual outcomes.

% ------------------------------------------------------------------------------
\subsubsection{Social Insurance}

Recognition that adverse outcomes often reflect bad luck rather than poor
choices supports comprehensive social insurance. Unemployment benefits, healthcare
access, and disability support become justified as protections against
uncontrollable risks rather than rewards for underperformance.

% ------------------------------------------------------------------------------
\subsubsection{Educational Policy}

Measuring luck in educational outcomes can inform admission policies and
resource allocation. If standardized test performance reflects
substantial circumstantial luck (parental resources, school quality, neighborhood
effects), policies that emphasize fine-grained score differences become less
defensible. Alternative approaches such as lottery-based admission among
qualified candidates warrant consideration.

% ==============================================================================
\subsection{Institutional Design}

Organizations can incorporate luck awareness in evaluation and reward systems.

% ------------------------------------------------------------------------------
\subsubsection{Performance Evaluation}

Corporations and public institutions that evaluate employees on outcomes
without accounting for luck risk misattributing results. Luck-adjusted performance
metrics provide fairer assessments by comparing actual outcomes to expected
outcomes given circumstances and constraints.

% ------------------------------------------------------------------------------
\subsubsection{Promotion and Selection}

When selecting among candidates with similar qualifications, acknowledging
that fine-grained differences may reflect luck rather than ability supports
lottery-based selection or increased weight on diverse criteria beyond narrow
performance measures.

% ------------------------------------------------------------------------------
\subsubsection{Executive Compensation}

CEO pay often reflects firm performance that depends substantially on
market conditions and timing. Luck-adjusted compensation schemes could separate
controllable performance from market-driven outcomes, reducing excessive
rewards for favorable circumstances.

% ==============================================================================
\subsection{Empirical Rankings of Activities and Occupations}

To illustrate the framework's application, we present empirical rankings
of common activities and occupations according to the relative
importance of luck versus skill in determining three distinct outcomes: income,
social status, and societal impact. These rankings demonstrate how the
skill-luck balance varies systematically across domains and outcome
types.

% ------------------------------------------------------------------------------
\subsubsection{General Performance Rankings}

Table~\ref{tab:general_rankings} provides a baseline ranking of activities
by their overall dependence on luck versus skill for typical performance
outcomes. Luck-dominated activities show little relationship between
individual effort and results. Mixed activities require skill but exhibit
high outcome variance from uncontrollable factors. Skill-dominated activities
show reliable skill-outcome relationships where ability and practice
predictably improve results. Rankings assume many trials over long careers;
at short timescales, luck matters more in almost all domains. Institutions
can amplify or suppress luck through safety nets, progressive structures,
or tournament designs. This general classification establishes patterns that
recur across specific outcome dimensions analyzed in subsequent tables.

\begin{table}[H]
  \centering
  \small
  \begin{tabular}{clll}
    \toprule \textbf{Rank} & \textbf{Activity / Job}               & \textbf{Dominant Factor} & \textbf{Notes}                              \\
    \midrule 1             & Lottery gambling                      & Luck                     & Outcomes almost entirely random             \\
    2                      & Casino games (slots, roulette)        & Luck                     & Skill has negligible impact                 \\
    3                      & Inheritance / family background       & Luck                     & Determined before individual action         \\
    4                      & Viral social media fame               & Luck                     & Timing and algorithms dominate              \\
    5                      & Acting / entertainment stardom        & Luck-heavy               & Talent necessary but not sufficient         \\
    6                      & Professional sports draft success     & Luck-heavy               & Talent + injuries + team context            \\
    7                      & Venture capital startup success       & Luck-heavy               & Market timing dominates outcomes            \\
    8                      & Stock picking (short-term trading)    & Luck-heavy               & Noise overwhelms signal short-term          \\
    9                      & Academic career (tenure-track)        & Mixed                    & Skill + timing + gatekeepers                \\
    10                     & Corporate executive promotion         & Mixed                    & Performance + politics + timing             \\
    11                     & Sales (high-variance environments)    & Mixed                    & Skill matters but randomness large          \\
    12                     & Entrepreneurship (small business)     & Mixed                    & Execution + location + shocks               \\
    13                     & Journalism / writing careers          & Mixed                    & Quality + attention dynamics                \\
    14                     & Software startup engineering          & Skill-leaning            & Skill important, market still risky         \\
    15                     & Competitive chess / esports           & Skill-leaning            & Luck minimal over many games                \\
    16                     & Professional programming              & Skill                    & Outcomes strongly skill-driven              \\
    17                     & Engineering (civil, mechanical)       & Skill                    & Errors and successes traceable              \\
    18                     & Medicine (clinical practice)          & Skill                    & High training, bounded uncertainty          \\
    19                     & Skilled trades (electrician, plumber) & Skill                    & Experience directly impacts results         \\
    20                     & Craftsmanship (luthier, watchmaker)   & Skill                    & Mastery dominates outcomes                  \\
    21                     & Mathematics / theoretical research    & Skill                    & Luck in insight exists, but skill dominates \\
    22                     & Classical music performance           & Skill                    & Precision and training decisive             \\
    \bottomrule
  \end{tabular}
  \caption{General ranking of activities by luck versus skill.}
  \label{tab:general_rankings}
\end{table}

This baseline categorization reveals several patterns. Pure gambling
activities sit at the extreme luck end, where outcomes are determined entirely
by chance mechanisms. Inheritance represents circumstantial luck
assigned before individual agency. Entertainment and venture capital occupy
a luck-heavy middle where talent is necessary but insufficient due to winner-take-all
dynamics and timing effects. Professional services like medicine, engineering,
and skilled trades cluster at the skill end where feedback loops, training,
and accumulated experience reliably improve outcomes. These patterns persist
across the specific outcome dimensions examined below.

% ------------------------------------------------------------------------------
\subsubsection{Income Rankings}

Table~\ref{tab:income_rankings} ranks activities by the degree to which
luck versus skill determines income outcomes. Several patterns emerge.
At the luck-dominated extreme, lottery gambling and casino games produce
outcomes almost entirely independent of player skill. Inheritance and
viral fame represent pure circumstantial luck. The middle range contains
mixed activities where skill is necessary but insufficient: acting, venture
capital, and stock trading all require ability but outcomes depend
heavily on timing, context, and random events. At the skill-dominated end,
craftsmanship, medicine, and mathematics show strong skill-outcome
relationships where ability and practice reliably improve results.

\begin{table}[H]
  \centering
  \small
  \begin{tabular}{clll}
    \toprule \textbf{Rank} & \textbf{Activity / Job}               & \textbf{Dominant Factor} & \textbf{Notes}                              \\
    \midrule 1             & Lottery gambling                      & Luck                     & Outcomes almost entirely random             \\
    2                      & Casino games (slots, roulette)        & Luck                     & Skill has negligible impact                 \\
    3                      & Inheritance / family background       & Luck                     & Determined before individual action         \\
    4                      & Viral social media fame               & Luck                     & Timing and algorithms dominate              \\
    5                      & Acting / entertainment stardom        & Luck-heavy               & Talent necessary but not sufficient         \\
    6                      & Professional sports draft success     & Luck-heavy               & Talent + injuries + team context            \\
    7                      & Venture capital startup success       & Luck-heavy               & Market timing dominates outcomes            \\
    8                      & Stock picking (short-term trading)    & Luck-heavy               & Noise overwhelms signal short-term          \\
    9                      & Academic career (tenure-track)        & Mixed                    & Skill + timing + gatekeepers                \\
    10                     & Corporate executive promotion         & Mixed                    & Performance + politics + timing             \\
    11                     & Sales (high-variance environments)    & Mixed                    & Skill matters but randomness large          \\
    12                     & Entrepreneurship (small business)     & Mixed                    & Execution + location + shocks               \\
    13                     & Journalism / writing careers          & Mixed                    & Quality + attention dynamics                \\
    14                     & Software startup engineering          & Skill-leaning            & Skill important, market still risky         \\
    15                     & Competitive chess / esports           & Skill-leaning            & Luck minimal over many games                \\
    16                     & Professional programming              & Skill                    & Outcomes strongly skill-driven              \\
    17                     & Engineering (civil, mechanical)       & Skill                    & Errors and successes traceable              \\
    18                     & Medicine (clinical practice)          & Skill                    & High training, bounded uncertainty          \\
    19                     & Skilled trades (electrician, plumber) & Skill                    & Experience directly impacts results         \\
    20                     & Craftsmanship (luthier, watchmaker)   & Skill                    & Mastery dominates outcomes                  \\
    21                     & Mathematics / theoretical research    & Skill                    & Luck in insight exists, but skill dominates \\
    22                     & Classical music performance           & Skill                    & Precision and training decisive             \\
    \bottomrule
  \end{tabular}
  \caption{Ranking of activities by luck versus skill in determining income.}
  \label{tab:income_rankings}
\end{table}

Three key patterns emerge from income analysis. First, income is more luck-driven
than performance in winner-take-all markets where visibility and timing dominate.
Second, skill dominates income in regulated, credentialed, or craft-based
professions with direct feedback loops. Third, the upper tail of income distributions
is almost always luck-amplified: even in skill-based fields, the highest
earners typically benefit from favorable circumstances beyond their control.

Important caveats apply. These rankings assume many trials (long careers,
repeated outcomes). At short timescales, luck matters more in almost all
domains. Institutions can amplify or suppress luck through safety nets, progressive
taxation, or tournament structures. Individual cases within any category
will vary.

% ------------------------------------------------------------------------------
\subsubsection{Status Rankings}

Table~\ref{tab:status_rankings} ranks activities by luck versus skill in
determining social status. Status shows greater luck-sensitivity than income.
Visibility and narrative dynamics dominate over contribution. Royalty and
celebrity status are almost entirely luck-determined, assigned by birth
or attention cascades. Even at the skill end, status depends
substantially on recognition mechanisms, institutional prestige, and
social networks—factors only partially controllable by individuals.

\begin{table}[H]
  \centering
  \small
  \begin{tabular}{clll}
    \toprule \textbf{Rank} & \textbf{Activity / Role}                  & \textbf{Status Driver} & \textbf{Notes}                   \\
    \midrule 1             & Royalty / inherited aristocracy           & Luck                   & Status assigned at birth         \\
    2                      & Celebrity by viral exposure               & Luck-heavy             & Attention cascades dominate      \\
    3                      & Influencer / public figure (online)       & Luck-heavy             & Algorithms and timing crucial    \\
    4                      & Entertainment stardom                     & Luck-heavy             & Winner-take-all recognition      \\
    5                      & Elite professional athlete                & Luck-heavy             & Skill required, exposure uneven  \\
    6                      & Billionaire entrepreneur                  & Luck-heavy             & Extreme skew in recognition      \\
    7                      & Top political leader                      & Mixed                  & Skill + timing + coalition luck  \\
    8                      & Corporate CEO (major firm)                & Mixed                  & Networks and succession timing   \\
    9                      & Prestigious academic (elite institutions) & Mixed                  & Reputation dynamics matter       \\
    10                     & Renowned artist / author                  & Mixed                  & Skill filtered by cultural luck  \\
    11                     & Judge / senior civil servant              & Skill-leaning          & Credentialed legitimacy          \\
    12                     & Senior physician                          & Skill-leaning          & Status tied to expertise         \\
    13                     & Lawyer (high standing)                    & Skill-leaning          & Reputation grows with competence \\
    14                     & University professor                      & Skill-leaning          & Long-run skill signal            \\
    15                     & Engineer (senior / chartered)             & Skill                  & Respect tied to reliability      \\
    16                     & Architect                                 & Skill                  & Skill and judgment visible       \\
    17                     & Skilled trades master                     & Skill                  & Local reputation-based status    \\
    18                     & Scientist (non-celebrity)                 & Skill                  & Recognition tracks contribution  \\
    19                     & Teacher (experienced)                     & Skill                  & Status from trust and service    \\
    20                     & Craftsperson                              & Skill                  & Mastery-based respect            \\
    \bottomrule
  \end{tabular}
  \caption{Ranking of activities by luck versus skill in determining social status.}
  \label{tab:status_rankings}
\end{table}

The status rankings reveal that recognition mechanisms introduce substantial
luck even when underlying performance is skill-based. Winner-take-all
dynamics amplify small random differences—an early break, fortuitous timing,
or algorithmic promotion—into large status disparities. Media visibility,
institutional prestige, and network effects create path dependence where
initial luck compounds over time.

% ------------------------------------------------------------------------------
\subsubsection{Impact Rankings}

Table~\ref{tab:impact_rankings} ranks activities by luck versus skill in
producing societal impact. Interestingly, sustained impact correlates
more strongly with skill than either income or status. Viral content creators
achieve visibility without lasting effect. Celebrity activism leverages
platforms but often lacks depth. In contrast, physicians, educators,
engineers, and public health officials produce measurable, cumulative
benefits through skill application over time.

\begin{table}[H]
  \centering
  \small
  \begin{tabular}{clll}
    \toprule \textbf{Rank} & \textbf{Activity / Role}                & \textbf{Impact Driver} & \textbf{Notes}                      \\
    \midrule 1             & Viral content creator                   & Luck                   & Impact unpredictable, fleeting      \\
    2                      & Celebrity activism                      & Luck-heavy             & Platform matters more than depth    \\
    3                      & Speculative finance                     & Luck-heavy             & Impact often indirect or neutral    \\
    4                      & Political leader (short tenure)         & Luck-heavy             & Context dominates effectiveness     \\
    5                      & Startup founder (median outcome)        & Mixed                  & Few high-impact successes           \\
    6                      & Journalist (agenda-setting roles)       & Mixed                  & Skill + institutional leverage      \\
    7                      & Policy-maker / regulator                & Mixed                  & Skill filtered through politics     \\
    8                      & Senior corporate executive              & Mixed                  & Impact depends on firm context      \\
    9                      & NGO leader                              & Skill-leaning          & Execution and governance matter     \\
    10                     & Urban planner                           & Skill-leaning          & Long-term structural effects        \\
    11                     & Public health official                  & Skill-leaning          & Expertise strongly affects outcomes \\
    12                     & Physician                               & Skill-leaning          & Direct, measurable human impact     \\
    13                     & Engineer (infrastructure)               & Skill                  & Safety and reliability critical     \\
    14                     & Scientist (applied research)            & Skill                  & Knowledge accumulation              \\
    15                     & Educator (systemic reach)               & Skill                  & Compounding long-term effects       \\
    16                     & Software engineer (widely used systems) & Skill                  & Scalable, repeatable impact         \\
    17                     & Environmental scientist                 & Skill                  & Policy-relevant evidence            \\
    18                     & Civil servant (operational roles)       & Skill                  & Consistent service delivery         \\
    19                     & Skilled trades (utilities, safety)      & Skill                  & Quiet but essential impact          \\
    20                     & Caregiver / nurse                       & Skill                  & High direct human impact            \\
    \bottomrule
  \end{tabular}
  \caption{Ranking of activities by luck versus skill in producing societal impact.}
  \label{tab:impact_rankings}
\end{table}

The impact rankings demonstrate a striking pattern: high-impact work is
often low-status and exhibits low variance. Caregivers, civil servants, and
skilled tradespeople produce consistent, essential benefits that
accumulate reliably over careers. Institutions with feedback loops—regulatory
oversight, professional standards, service delivery metrics—reduce luck's
role in impact generation. In contrast, activities that produce high status
or income through luck rarely generate sustained societal benefit.

These rankings illustrate the framework's analytical power. By
decomposing outcomes into skill and luck components, we can identify systematic
patterns: winner-take-all markets amplify luck, credentialed professions
reduce it, and sustained impact requires skill regardless of status or
income. Such analysis informs both individual career decisions and institutional
design aimed at aligning rewards with contributions.

% ------------------------------------------------------------------------------
\subsubsection{Combined Luck Dependence and Social Impact}

Table~\ref{tab:combined_rankings} synthesizes luck dependence with
social impact to reveal which activities produce reliable, sustained benefits
versus fleeting effects driven by chance. The table is ordered from high luck /
low reliable impact to low luck / high reliable impact. Luck dependence
measures the extent to which outcomes vary due to randomness, timing, or
attention rather than skill. Social impact measures the durability, scale, and
reliability of positive effects on others. This combined perspective highlights
a fundamental tension: activities yielding high visibility and personal rewards
often produce minimal lasting impact, while low-visibility work with strong
skill-outcome relationships generates consistent societal value. High-luck
roles can achieve huge impact, but only rarely and unpredictably. Low-luck
roles deliver consistent, compounding impact across time. The inverse
relationship suggests that activities with greatest reliable social impact are
typically those where luck matters least and feedback, standards, and
accountability are strongest.

\begin{table}[H]
  \centering
  \small
  \begin{tabular}{clccc}
    \toprule \textbf{Rank} & \textbf{Activity / Role}                & \textbf{Luck}       & \textbf{Impact}    & \textbf{Notes}                    \\
                           &                                         & \textbf{Dependence} & \textbf{(Typical)} &                                   \\
    \midrule 1             & Lottery winner / viral celebrity        & Very High           & Very Low           & Visibility without durable impact \\
    2                      & Viral content creator                   & Very High           & Low                & Impact short-lived, unpredictable \\
    3                      & Celebrity activism                      & High                & Low--Medium        & Platform $>$ substance            \\
    4                      & Speculative finance / trading           & High                & Low                & Redistribution, little net value  \\
    5                      & Entertainment stardom                   & High                & Low--Medium        & Cultural impact uneven            \\
    6                      & Political leader (short tenure)         & High                & Medium             & Context dominates outcomes        \\
    7                      & Venture-backed startup founder (median) & High                & Medium             & Few outsized successes            \\
    8                      & Influencer entrepreneur                 & High                & Medium             & Impact depends on audience luck   \\
    9                      & Journalist (agenda-setting roles)       & Medium              & Medium             & Skill + institutional leverage    \\
    10                     & Corporate executive (large firm)        & Medium              & Medium             & Impact constrained by system      \\
    11                     & NGO leader                              & Medium              & Medium--High       & Execution skill matters           \\
    12                     & Policymaker / regulator                 & Medium              & Medium--High       & Skill filtered through politics   \\
    13                     & Urban planner                           & Low--Medium         & High               & Long-term structural effects      \\
    14                     & Public health official                  & Low--Medium         & High               & Evidence-driven outcomes          \\
    15                     & Educator (system-level reach)           & Low--Medium         & High               & Compounding long-term impact      \\
    16                     & Scientist (applied research)            & Low                 & High               & Knowledge accumulation            \\
    17                     & Software engineer (widely used systems) & Low                 & High               & Scalable, repeatable impact       \\
    18                     & Engineer (infrastructure, safety)       & Low                 & Very High          & Failure intolerant                \\
    19                     & Civil servant (operations)              & Low                 & Very High          & Reliability over visibility       \\
    20                     & Nurse / caregiver                       & Very Low            & Very High          & Direct human impact               \\
    21                     & Skilled trades (utilities, safety)      & Very Low            & Very High          & Quiet, essential services         \\
    22                     & Physician                               & Very Low            & Very High          & Skill tightly linked to outcomes  \\
    \bottomrule
  \end{tabular}
  \caption{Combined ranking by luck dependence and social impact.}
  \label{tab:combined_rankings}
\end{table}

The combined rankings reveal a striking inverse relationship. At ranks 1--8,
high luck dependence correlates with low or fleeting impact. Viral celebrities
and content creators achieve visibility through algorithmic chance but produce
minimal lasting value. Entertainment stardom and speculative finance
offer high personal rewards but uncertain societal benefits. The middle ranks
(9--15) show mixed patterns where institutional context mediates between
skill and impact. Journalists, policymakers, and NGO leaders operate in environments
where execution skill matters but political and organizational luck filters
effectiveness.

At ranks 16--22, low luck dependence correlates strongly with high sustained
impact. Physicians, nurses, engineers, and skilled tradespeople work in domains
with tight feedback loops, professional standards, and accountability mechanisms.
Their skills translate reliably into measurable benefits: infrastructure
safety, healthcare outcomes, essential services. This work often
receives low public visibility and modest status relative to its impact,
yet produces the consistent, compounding benefits that constitute
societal welfare.

This pattern suggests a fundamental misalignment between recognition
systems and value creation. Winner-take-all attention markets reward luck-driven
outcomes with disproportionate status and income, while skill-intensive
work producing reliable impact receives limited recognition. Understanding
this misalignment informs institutional design: societies might benefit
from mechanisms that reduce luck amplification in reward structures and
better recognize sustained, skill-driven contributions to collective welfare.

% ==============================================================================
\subsection{Cultural and Psychological Effects}

Wider recognition of luck's role may produce several cultural shifts:

\begin{itemize}
  \item \textbf{Increased Humility:} Successful individuals may develop
    greater appreciation for favorable circumstances rather than
    attributing outcomes entirely to personal merit.

  \item \textbf{Reduced Stigma:} Failure may be viewed with less moral
    judgment when understood as partially reflecting bad luck.

  \item \textbf{Systems Thinking:} Cultural narratives may shift from
    individual hero stories toward recognition of structural conditions
    and systemic factors.

  \item \textbf{Risk of Fatalism:} Overemphasis on luck without careful
    communication could reduce motivation and effort, particularly in educational
    contexts.
\end{itemize}

% ==============================================================================
\subsection{Limitations and Caveats}

Several limitations warrant acknowledgment. First, quantifying control
remains challenging and often requires judgment calls. Second, the
framework assumes agents have definable utility functions, which may not
capture all relevant dimensions of value. Third, practical application requires
substantial data that may not always be available. Fourth, emphasizing luck
could produce unintended psychological effects if not communicated carefully.

Despite these limitations, the framework provides a structured approach
to a question that otherwise remains informal and speculative.

% ##############################################################################
\section{Related Work}

Recent research has advanced the formal quantification and empirical analysis
of skill and luck across multiple domains. This work complements and
extends our theoretical framework by providing domain-specific
applications and empirical validation.

\textbf{Game-Theoretic Foundations.} Silver~\cite{silver2025} introduces
a formal index to disentangle skill and luck in stochastic games by decomposing
outcomes into ``skill leverage'' and ``luck leverage.'' This framework
provides algorithms to compute these components from game trees and maps
games on a spectrum from pure chance to pure skill. The approach generalizes
beyond games to any decision process under uncertainty, with
applications in game design, AI performance evaluation, and risk
management.

Jerdee and Newman~\cite{jerdee2024} extend classical ranking models by
adding a ``luck'' parameter capturing random upsets and a ``depth of
competition'' metric. Fitting this model to sports, games, and social hierarchies,
they find that human social hierarchies tend to be deep with significant
luck factors, while sports and games show shallow competition with minimal
luck-driven upsets. This quantification demonstrates different luck-skill
balances across domains.

\textbf{Normative and Philosophical Analysis.} Liu and Tsay~\cite{liu2023}
build on prior chance models to formalize when high performance misleadingly
signals merit versus luck. They define four versions of a normative luck
framework integrating psychological insights with decision models, predicting
conditions under which extreme success may indicate greater luck and even
lower expected skill. They illustrate how highly cited publications can
sometimes reflect lower true research quality through chance success, and
discuss strategies to correct biases that lead people to mistake luck for
skill.

Lefranc and Trannoy~\cite{lefranc2025} situate the skill-luck distinction
in inequality and social justice contexts. They refine equal opportunity
definitions by distinguishing how luck enters before or after effort. Their
analysis highlights that the timing of luck relative to decision-making critically
affects how policy should compensate for or nullify luck's effects.

\textbf{Agent-Based and Simulation Models.} Pluchino, Biondo, and
Rapisarda~\cite{pluchino2018} use agent-based modeling to explore how
random events combined with talent distributions produce outcome
distributions. They show that multiplicative effects of random lucky events
can yield extremely skewed success outcomes even with normally distributed
talent. Greatest successes often require being fortunate multiple times
rather than being most talented, providing quantitative basis for the meritocracy-versus-luck
debate.

\textbf{Empirical Studies in Education and Labor Markets.} Landaud et al.~\cite{landaud2022}
exploit a natural experiment in Norwegian high school exams to measure luck.
Students randomly assigned exam subjects experience ``lucky'' outcomes
when tested in their strongest subjects. This exam luck significantly boosts
grades, graduation probability, and leads to substantial lasting wage differences
comparable to effects of parental education or teacher quality,
underlining luck's role in educational and labor outcomes.

\textbf{Applications in Sports.} Holzmeister and Johannesson~\cite{holzmeister2025}
quantify skill versus luck in professional soccer using seven European leagues.
They decompose team performance relative to expected goals into skill
and luck components, estimating that approximately 40\% of over-or-under-performance
variation is skill while 60\% is luck. Simulations show luck changes the
champion in 34\% of seasons and decides relegations in 76\% of cases, demonstrating
that chance significantly sways season outcomes alongside skill.

\textbf{Corporate Governance and Executive Compensation.} Al-Sabah~\cite{alsabah2020}
examines whether CEOs accumulate influence through skill or fortunate
outcomes. Findings show that good luck significantly increases CEO power
within firms, while measurable skill has smaller effects. Boards often inadvertently
reward CEOs for lucky outcomes, expanding authority beyond what skill
alone would merit.

Bertrand and Mullainathan~\cite{bertrand2001} define luck as observable
shocks to firm performance outside CEO control and test whether such
luck affects compensation. CEO pay increases due to lucky outcomes as much
as for skill-driven performance. However, firms with strong governance better
filter out luck, granting smaller raises for luck-driven gains. This evidence
of ``pay-for-luck'' sparked debates on fair compensation and incentive contract
design.

\textbf{Financial Markets.} Fama and French~\cite{fama2010} address
whether outperforming investment fund managers are skillful or lucky. Using
bootstrap simulation on decades of U.S. mutual fund data, they show very
few funds earn returns beyond chance expectations after fees. Most
apparent outperformance can be explained by luck, reinforcing caution in
attributing short-term success to manager skill.

Each of these works either builds upon similar luck-skill decomposition themes
or applies them across domains including games, sports, education,
finance, and corporate governance. Together, they advance theoretical foundations
and provide empirical demonstrations, expanding our ability to quantify
and act upon the interplay of skill and luck in real-world outcomes.

% ##############################################################################
\section{Conclusion}
\label{sec:conclusion}

This paper develops a formal framework for quantifying skill and luck in
agent outcomes. By integrating concepts from probability theory, decision
theory, and causal inference, the approach provides mathematically precise
definitions and measurement procedures. The framework distinguishes
three essential components—probability, utility, and control—and
combines them in a multiplicative luck function that satisfies intuitive
axioms.

Applications span individual event rating, performance evaluation, and
societal-level analysis. At the individual level, the framework enables
luck-adjusted assessment of outcomes. At the aggregate level, it
supports measurement of luck inequality, circumstantial dependence, and
social mobility. These measurements inform debates about fairness,
desert, and policy design.

Several directions for future work appear promising. Empirical
applications to specific domains such as labor markets, financial markets,
or educational outcomes could validate the framework and refine
estimation methods. Extensions incorporating dynamic considerations such
as path dependence and cumulative advantage would enhance understanding of
lifetime trajectories. Integration with behavioral economics and
psychology could address perception biases and motivation effects. Normative
analysis of optimal policy under measured luck distributions could translate
descriptive findings into prescriptive recommendations.

The framework suggests that outcomes depend more substantially on luck than
commonly acknowledged. This recognition has implications for how
societies structure incentives, allocate resources, and make judgments
about individual responsibility. By providing formal tools for measurement
and analysis, the framework enables more rigorous and evidence-based
engagement with these questions.