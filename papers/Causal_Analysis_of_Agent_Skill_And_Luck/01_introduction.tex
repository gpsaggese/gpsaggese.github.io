\title{A Causal Analysis of Skill and Luck in Agent Outcomes}

\author{Giacinto Paolo Saggese$^{*}$}
\thanks{$^{*}$ Causify.AI, The University of Maryland, College Park, MD 20742,
USA, gsaggese@umd.edu}

\date{\today}

\maketitle

\begin{abstract}
  The inability to rigorously separate skill from luck in human outcomes
  represents perhaps the most consequential unsolved problem in social science,
  underpinning every question of justice, merit, responsibility, and institutional
  design. Despite millennia of philosophical inquiry and centuries of
  statistical analysis, we have lacked the mathematical foundation to quantify
  what portion of observed inequality reflects genuine differences in ability
  versus the capricious distribution of fortune. This paper provides
  that foundation.

  We develop the first complete mathematical framework for decomposing
  agent outcomes into skill and luck components by integrating
  probability theory, decision theory, and causal inference. Our formalization
  defines luck as value-weighted surprise of outcomes not explained by controllable
  factors, enabling principled measurement of both individual events and
  population-level distributions. This framework fundamentally transforms
  our ability to address questions that have remained intractable: How
  much of economic inequality is attributable to circumstances versus
  choices? What degree of outcome variation lies beyond individual control?
  How should institutions account for luck when evaluating performance
  and allocating resources?

  The implications are profound and far-reaching. This framework provides
  the missing analytical foundation for understanding social stratification,
  meritocracy, and distributive justice. It enables evidence-based
  policy design by quantifying the extent to which outcomes reflect
  factors beyond individual agency. It offers a rigorous basis for moral
  philosophy's treatment of responsibility and desert. And it supplies
  the tools necessary for evaluating whether our institutions reward
  skill or merely reinforce the accidents of birth and circumstance. By making
  measurable what was previously only intuited, this work establishes
  the theoretical basis for a more scientifically grounded understanding
  of the human condition in modern society.
\end{abstract}

\setcounter{tocdepth}{1}
\tableofcontents

% ##############################################################################
\section{Introduction}

The attribution of outcomes to skill versus luck influences judgments across
domains including moral philosophy, economic policy, educational assessment,
and institutional design. Despite widespread intuition that both factors
matter, formal methods for separating and quantifying these
contributions remain underdeveloped. This gap impedes rigorous analysis of
questions such as: To what extent do observed inequalities reflect
differences in ability versus circumstance? How should institutions account
for luck when evaluating performance? What policies are justified when
outcomes depend substantially on factors beyond individual control?

Existing approaches often conflate luck with randomness or treat it as a
residual category. We propose that luck can be formally defined and measured
through a multi-component framework that integrates probability (rarity
of events), utility (value to the agent), and control (degree of agency
over outcomes). This formalization enables quantitative analysis while
preserving philosophical distinctions between different forms of luck.

The framework addresses three interconnected problems. First, it
provides a method for rating individual events on a luck scale, accounting
for context-dependent information and agent-specific values. Second, it enables
decomposition of outcomes into skill and luck components through counterfactual
reasoning and expected value calculations. Third, it supports societal-level
analysis by aggregating individual luck scores and measuring distributional
properties such as luck inequality and the correlation between circumstances
and outcomes.

This paper proceeds as follows. Section~\ref{sec:contributions}
enumerates our major contributions. Section~\ref{sec:background} discusses
the conceptual foundations and motivation. Section~\ref{sec:theory}
develops the mathematical theory, including axioms, probability models, and
the core luck function. Section~\ref{sec:rating} presents practical
methods for rating events and decomposing skill from luck. Section~\ref{sec:society}
extends the framework to societal modeling. Section~\ref{sec:implications}
examines consequences for policy and institutions. Section~\ref{sec:conclusion}
concludes.

% ##############################################################################
\section{Contributions}
\label{sec:contributions}

This paper makes the following major contributions:

\begin{itemize}
  \item \textbf{First Complete Mathematical Framework:} We provide the
    first rigorous formalization integrating probability theory, decision
    theory, and causal inference to decompose outcomes into skill and
    luck components.

  \item \textbf{Axiomatic Foundation:} We establish axioms ensuring
    theoretical soundness, including scale invariance, additivity, and proper
    control attribution.

  \item \textbf{Unified Cross-Disciplinary Theory:} Our framework
    bridges moral philosophy, economics, statistics, and causal inference,
    resolving conceptual ambiguities across disciplines.

  \item \textbf{Practical Measurement Methodology:} We develop
    implementable methods for rating events and computing luck scores
    from observational or experimental data.

  \item \textbf{Societal-Level Analysis Framework:} We extend from
    individual events to population dynamics, enabling measurement of luck
    inequality and empirical evaluation of meritocracy.

  \item \textbf{Causal Model of Outcomes:} We formalize relationships
    between circumstances, choices, and shocks, enabling variance decomposition
    and policy analysis.

  \item \textbf{Evidence-Based Foundation for Policy:} By making luck
    measurable, we transform philosophical debates about responsibility and
    justice into empirically tractable questions.
\end{itemize}

These contributions establish the theoretical foundation for a scientific
approach to merit, fairness, and responsibility.

% ##############################################################################
\section{Background and Motivation}
\label{sec:background}

% ==============================================================================
\subsection{Conceptual Foundations}

The concept of luck appears in diverse contexts with varying definitions.
In ordinary language, luck refers to outcomes influenced by chance. In
moral philosophy, particularly the literature on moral luck, it denotes
factors outside an agent's control that affect judgments of praise or
blame. In statistical analysis, luck often represents deviations from
expected performance or unexplained variance.

We propose a unifying definition: luck is the value-weighted surprise of
an outcome that the agent did not significantly control, relative to the
agent's prior information. This definition incorporates three essential components:

\begin{itemize}
  \item \textbf{Probability:} Low-probability events contribute more to
    luck than expected occurrences.

  \item \textbf{Value:} Events must matter to the agent; neutral outcomes
    are not lucky.

  \item \textbf{Control:} Outcomes predominantly caused by the agent's
    choices reflect skill rather than luck.
\end{itemize}

These components distinguish luck from related concepts. Pure randomness
becomes luck only when it affects valued outcomes. Skill represents the
portion of outcomes explained by controllable actions. Circumstance refers
to background conditions that shape probabilities and opportunities.
Figure~\ref{fig:components} illustrates how these three components combine
multiplicatively in our framework: probability is transformed into surprise
(rarer events score higher), control is inverted (less control increases luck),
and utility provides value weighting.

The distinction between luck and skill manifests through multiple related
conceptual pairs. Table~\ref{tab:terminology} summarizes common
terminology across disciplines, with external factors (luck) on the left
and internal factors (skill) on the right. External factors represent
uncontrollable influences on outcomes, while internal factors represent
controllable determinants. These pairs reflect different
emphases—philosophical (brute luck vs. earned skill), statistical (noise
vs. signal), causal (exogenous vs. endogenous), or temporal (timing vs.
judgment)—but all capture the fundamental dichotomy between controllable
and uncontrollable outcome determinants.

\begin{table}[H]
  \centering
  \begin{tabular}{ll}
    \toprule \textbf{Luck / External Factors} & \textbf{Skill / Internal Factors} \\
    \midrule Luck                             & Skill                             \\
    Chance                                    & Ability                           \\
    Randomness                                & Competence                        \\
    Variance                                  & Consistency                       \\
    Noise                                     & Signal                            \\
    Fortune                                   & Merit                             \\
    Happenstance                              & Expertise                         \\
    Contingency                               & Mastery                           \\
    Exogenous factors                         & Endogenous factors                \\
    Brute luck                                & Earned skill                      \\
    Random shocks                             & Preparation                       \\
    Timing                                    & Judgment                          \\
    Opportunity                               & Capability                        \\
    \bottomrule
  \end{tabular}
  \caption{Terminology for luck versus skill across disciplines.}
  \label{tab:terminology}
\end{table}

\begin{figure}[H]
  \centering
  \begin{tikzpicture}[
    component/.style={rectangle, draw, fill=blue!20, minimum width=2.5cm, minimum height=1cm, align=center},
    multiplier/.style={circle, draw, fill=gray!20, minimum size=0.8cm},
    arrow/.style={->, >=stealth, thick}
  ]
    % Three components
    \node[component] (prob) at (0,0) {Probability\\$P(E|I_{A})$};
    \node[component] (util) at (0,-2) {Utility\\$U_{A}(E)$};
    \node[component] (ctrl) at (0,-4) {Control\\$C_{A}(E)$};

    % Transformations
    \node[component, fill=green!20]
      (surp)
      at
      (4,0)
      {Surprise\\$-\log P(E|I_{A})$};
    \node[component, fill=orange!20]
      (lack)
      at
      (4,-4)
      {Lack of Control\\$(1-C_{A}(E))$};

    % Result
    \node[multiplier] (mult1) at (8,-1) {$\times$};
    \node[multiplier] (mult2) at (8,-3) {$\times$};
    \node[component, fill=red!30, minimum width=3cm]
      (luck)
      at
      (11,-2)
      {Luck Score\\$L_{A}(E)$};

    % Arrows
    \draw[arrow] (prob) -- (surp);
    \draw[arrow] (ctrl) -- (lack);
    \draw[arrow] (surp) -- (mult1);
    \draw[arrow] (util) -- (mult1);
    \draw[arrow] (mult1) -- (mult2);
    \draw[arrow] (lack) -- (mult2);
    \draw[arrow] (mult2) -- (luck);

    % Labels
    \node[above=0.1cm of surp, font=\small] {rare $\rightarrow$ high};
    \node[above=0.1cm of lack, font=\small]
      {less control $\rightarrow$ high};
  \end{tikzpicture}
  \caption{Components of the luck function.}
  \label{fig:components}
\end{figure}

% ==============================================================================
\subsection{Motivation for Formalization}

Formalizing luck serves multiple purposes. In moral and political
philosophy, it clarifies debates about desert and responsibility. If
outcomes depend substantially on luck, claims that individuals deserve their
positions become more difficult to sustain. In institutional design, understanding
luck enables fairer evaluation systems that account for factors beyond individual
control. In policy analysis, measuring the role of luck informs debates about
taxation, social insurance, and equality of opportunity.

Empirical measurement requires precise definitions. Without formal
structure, assertions about the importance of luck remain speculative. A
mathematical framework enables testable predictions, quantitative comparisons
across contexts, and evidence-based policy recommendations.

% ==============================================================================
\subsection{Related Concepts}

Several mathematical and philosophical traditions inform this framework:

\begin{itemize}
  \item \textbf{Probability Theory:} Provides the foundation for
    quantifying surprise and rare events through probability measures
    and information theory.

  \item \textbf{Decision Theory:} Supplies utility functions for valuing
    outcomes and expected utility calculations for separating realized
    outcomes from expectations.

  \item \textbf{Causal Inference:} Offers tools for identifying
    controllable versus uncontrollable factors through causal graphs and
    counterfactual reasoning.

  \item \textbf{Game Theory:} Distinguishes strategic skill from chance
    elements in mixed games of skill and luck.

  \item \textbf{Moral Philosophy:} Examines the normative implications
    of luck through analyses of moral luck and distributive justice.
\end{itemize}