\documentclass[11pt, reqno]{amsart}
\usepackage{amsfonts, amssymb, amscd}
%\usepackage{amsrefs}
\usepackage{graphicx}
\usepackage{hyperref}
\usepackage{slashed}
\usepackage{fullpage}
% Prevent table repositioning.
\usepackage{float}
% For textcolor.
\usepackage{xcolor}
% For blackboard bold `1`.
\usepackage{bbold}

\usepackage{booktabs}
\usepackage{tabularx}
\usepackage{longtable, array}

\usepackage{algorithm}
\usepackage{algpseudocode}

\usepackage{makecell}

%\usepackage{titlesec}

%\titleformat*{\section}{\LARGE\bfseries}
%\titleformat*{\subsection}{\Large\bfseries}
%\titleformat*{\subsubsection}{\large\bfseries}
%\titleformat*{\paragraph}{\large\bfseries}
%\titleformat*{\subparagraph}{\large\bfseries}

% If you need math, theorems, etc.
\usepackage{amsmath, amssymb, amsthm}
\usepackage{tikz}
\usetikzlibrary{
  positioning,
  calc,
  arrows.meta,
  decorations.pathreplacing
}

\usepackage[useregional]{datetime2}
\DTMlangsetup[en-US]{zone=eastern,mapzone}

% --- Theorem environments (acmart loads amsthm; this is safe) ---
\theoremstyle{plain}
\newtheorem{theorem}{Theorem}[section]
\newtheorem{lemma}[theorem]{Lemma}
\newtheorem{proposition}[theorem]{Proposition}
\newtheorem{corollary}[theorem]{Corollary}

\theoremstyle{definition}
\newtheorem{definition}[theorem]{Definition}

\theoremstyle{remark}
\newtheorem{remark}[theorem]{Remark}

\newtheorem{assumption}{Assumption}[section]

\input{./helpers_root/dev_scripts_helpers/documentation/latex_abbrevs.sty}

% Python style for highlighting
\usepackage{listings}

\newcommand{\pythonstyle}{\lstset{ language=Python, basicstyle=\ttm, morekeywords={self}, % Add keywords here
keywordstyle=\ttb
\color{deepblue}
, emph={MyClass,__init__}, % Custom highlighting
emphstyle=\ttb
\color{deepred}
, % Custom highlighting style
stringstyle=
\color{deepgreen}
, frame=tb, % Any extra options here
showstringspaces=false }}

% Python environment.
\lstnewenvironment{python}[1][]{ \pythonstyle \lstset{#1} }{}

% Python for external files.
\newcommand{\pythonexternal}[2][]{{ \pythonstyle \lstinputlisting[#1]{#2}}}

% Python for inline.
\newcommand{\pythoninline}[1]{{\pythonstyle\lstinline!#1!}}

\usepackage{listings}
\usepackage{xcolor}

\definecolor{codegreen}{rgb}{0,0.6,0}
\definecolor{codegray}{rgb}{0.5,0.5,0.5}
\definecolor{codepurple}{rgb}{0.58,0,0.82}
\definecolor{backcolour}{rgb}{0.95,0.95,0.92}

\lstdefinestyle{mystyle}{ backgroundcolor=
\color{backcolour}
, commentstyle=
\color{codegreen}
, keywordstyle=
\color{magenta}
, numberstyle=\tiny
\color{codegray}
, stringstyle=
\color{codepurple}
, basicstyle=\ttfamily\footnotesize, breakatwhitespace=false, breaklines=true, captionpos=b,
keepspaces=true, numbers=left, numbersep=5pt, showspaces=false, showstringspaces=false,
showtabs=false, tabsize=2 }

\lstset{style=mystyle}

% end python

%

\providecommand{\tightlist}{%
\setlength{\itemsep}{0pt}
\setlength{\parskip}{0pt}}

% Use this if using `\contrib[]{...}`.
%\makeatletter\let\@wraptoccontribs\wraptoccontribs\makeatother
% https://tex.stackexchange.com/questions/418547/equal-contribution-using-thanks-with-llncs-class#418563
\makeatletter
\newcommand{\printfnsymbol}[1]{%
\textsuperscript{\@fnsymbol{#1}}%
}
\makeatother
